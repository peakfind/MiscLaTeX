% freefem_def.tex - language def for freefem when using listings
\documentclass{article}

\usepackage[bottom=2.5cm,top=2.5cm,right=3cm,left=3cm]{geometry}

% Define some colors for code
\usepackage{xcolor}
\definecolor{brickred}{RGB}{147,39,44}
\definecolor{darkgreen}{RGB}{0,103,0}
\usepackage{listings}

\lstdefinelanguage{freefem}{
  sensitive    = true,
  morecomment  = [l][\color{teal}]{//},
  morecomment  = [s][\color{teal}]{/*}{*/},
  keywords     = [2]{% Standard types 
  	                 int,bool,real,complex,string,
  	                 % Mesh design
  	                 border,mesh,mesh3,
  	                 % Finite element space design
  	                 fespace,
  	                 % Marco design
  	                 marco,
  	                 % Function design
  	                 func,
  	                 % Problem design
  	                 problem,solve,varf,
  	                 % array
  	                 matrix,P1},
  keywordstyle = [2]{\color{brickred}},
  keywords     = [3]{%Functions
                     abs,acos,acosh,adaptmesh,adj,AffineCG,
                     AffineGMRES,arg,asin,asinh,assert,atan,
                     atan2,atanh,atoi,atof,BFGS,buildmesh,ceil,
                     change,checkmovemesh,chi,clock,complexEigenValue,
                     conj,convect,copysign,cos,cosh,diffnp,diffpos,
                     dist,dumptable,dx,dxx,dxy,dxz,dy,dyx,dyy,dyz,dz,
                     dzx,dzy,dzz,EigenValue,emptymesh,erf,erfc,exec,
                     exit,exp,fdim,floor,fmax,fmin,fmod,imag,int1d,
                     int2d,int3d,intalledges,intallfaces,interpolate,
                     invdiff,invdiffnp,invdiffpos,isInf,isNaN,isNormal,j0,
                     j1,jn,jump,LinearCG,LinearGMRES,lgamma,log,log10,
                     lrint,lround,ltime,max,min,movemesh,NaN,NLCG,
                     on,plot,polar,pow,projection,randinit,randint31,
                     randint32,randreal1,randreal2,randreal3,randres53,
                     readmesh,readmesh3,rint,round,savemesh,set,sign,
                     signbit,sin,sinh,sort,splitmesh,sqrt,square,storagetotal,
                     storageused,strtod,strtol,swap,system,tan,tanh,tgamma,
                     time,trace,trunc,y0,y1,yn},
  keywordstyle = [3]{\color{blue}},
  keywords     = [4]{%Global variables
  	                 area,ARGV,BoundaryEdge,CG,Cholesky,Crout,
  	                 edgeOrientation,false,GMRES,hTriangle,
  	                 include,internalEdge,label,lenEdge,load,
  	                 LU,N,nTonEdge,nuEdge,nuTriangle,P,pi,
  	                 region,sparesolver,true,verbosity,version,
  	                 volume,x,y,z,fill,wait,periodic,fixedborder,
                   	 %Loops
                     for,if,else,while,continue,break,try,
                     catch,
                     %IO
                     cout,cin,endl,ifstream,ofstream,append,binary,
                     seekg,tellg,flush,getline,
                     %Output format
                     precision,scientific,fixed,showbase,
                     noshowbase,showpos,noshowpos,default,
                     setw},
  keywordstyle = [4]{\bfseries\color{darkgreen}},
 }
\lstset{
  language         = freefem,
  basicstyle       = \ttfamily,
  numbers          = left,
  numberstyle      = \tiny,
  columns          = fullflexible,keepspaces,
  showstringspaces = false,
}

\title{Listings Setting for FreeFEM code}
\author{Zhang Jiayi}
\date{\today}

\begin{document}
\maketitle

\begin{lstlisting}
/*
 * quasi.edp
 */
            
/*---------------------------------------------------------
 * Parameters & Functions
 *-------------------------------------------------------*/
real ksquare   = 1;     // coefficient k^2
real alpha     = 0.1;
complex lambda = 1i;    // coefficient of the impedance boundary condition
complex iunit  = 1i;
func f         = x * y; // right hand side for the Helmholtz equation
// the x component of outer normal vector
func nu1       = cos(x) / sqrt(cos(x) * cos(x) + 1); 
    
/*---------------------------------------------------------
 * Triangulation
 *-------------------------------------------------------*/
    
// Define mesh boundary
border Gamma1(t=0, 2){x=2*pi; y=t; label=1;}
border Gammab(t=0, 2*pi){x=2*pi - t; y=2; label=2;}
border Gamma2(t=0, 2){x=0; y=2 - t; label=3;}
border Gamma(t=0, 2*pi){x=t; y=sin(t); label=4;}
    
// show the boundary
plot(Gamma1(10) + Gammab(20) + Gamma2(10) + Gamma(40), wait=true); 
mesh Th = buildmesh(Gamma1(10) + Gammab(20) + Gamma2(10) + Gamma(40), fixedborder=true);
    
// show the triangulation
plot(Th, wait=true); 
            
/*---------------------------------------------------------
 * FEspace
 *-------------------------------------------------------*/
fespace Vh(Th, P1, periodic=[[1, y], [3, y]]);
Vh<complex> u, v;
// real part and imagnary part of the solution u
Vh realu, imagu; 
            
solve Problem(u, v) = int2d(Th)(dx(u) * dx(v) + dy(u) * dy(v))
                      + int2d(Th)(iunit * alpha * u * dx(v) - iunit * alpha * dx(u) * v)
                      - int2d(Th)((ksquare - square(alpha)) * u * v)
                      + int1d(Th, Gamma)(lambda * u * v + iunit * nu1 * alpha * u * v)
                      + int2d(Th)(exp(-iunit * alpha * x) * f * v)
                      + on(Gammab, u = 0);
            
realu = real(u);
imagu = imag(u);
            
// show the result
plot(realu, fill=true, wait=true);
plot(imagu, fill=true, wait=true);
\end{lstlisting}

\end{document}
